\section{Methodology}
\subsection{Data Extraction}
\noindent The data extraction has been performed on March 6, 2016, and included data from November 2013 to February 2016 from the Stack Exchange (SE) data dump\footnote{https://archive.org/details/stackexchange} and for this study, we considered the Portuguese version (SOPT). The XML files corresponding to the tags, users, and posts were transferred to a MySql database, through a R function per type of file (i.e., posts, users, and tags). 
\subsection{Data Preprocesing}
We cleansed the tables eliminating few users were due to lack of data. None of these users had AccountId (i.e., user identifier for all stackExchange websites), LastAccessDate, WebsiteUrl, Location, UpVotes, DownVotes or Age. All of these users have the same display name (i.e., "a25bedc5-3d09-41b8-82fb-ea6c353d75ae"), and whenever they have a ProfileImageUrl, it is the same\footnote{https://www.gravatar.com/avatar/?s=128\&d=identicon\&r=PG\&f=1}. These accounts were created at different times from November 2015 to February 2016. We could not come up with a plausible reason for these anonymous users having the same display name but no other data, they do not seem to have anything in common. In total 3 SOPT users have been eliminated. 

To identify their location we used \texttt{countryNameManager}\footnote{https://github.com/tue-mdse/countryNameManager}.

\subsection{Searching Process}
Consequently, the locations were identified, and before starting the search a group of students were selected based on whether they had experience in use SO (in Spanish version) or GitHub, as detailed below:\\
The search started doing a manual inspection by each user profile based on \textit{userId}, i.e. \url{http://pt.stackoverflow.com/users/1919/}, where \textbf{1919} is the \textit{userId}. On the user profile, we looked the email address, if it was not available, we checked whether the user has a GitHub account or a personal web page. 
        \begin{itemize}
            \item With Github account we found out the email address below \textit{userName} if it was not on, we used a browser extension gitDiscovered\footnote{https://gitdiscovered.com/} to discover the email address or we checked his/her public activity looking for at he/she did a git command \footnote{https://git-scm.com/docs/git-push}, and then we searched the email address using a Github API \footnote{https://api.github.com/users/userName/events/public} by \textit{userName}.
            \item With the personal web page, we searched the email address on the section "about me" or \textit{sobre me} in the Portuguese language.
            \item If none of the above, we used GitHub and searched by userName from SOPT, and compared profile picture, location, skills between SOPT and search results on GitHub, and then we followed the steps above mentioned with GitHub account, in order to find the user and get the email address.              
       \end{itemize}            
 
In order to ensure the accuracy of results, we selected a random group of 15 users each 500 profiles and searched using the steps above mentioned. Whether we found new email addresses or missing information, we chose another one random group and applied the manual inspection again.

