\section{Methodology}
\noindent To conduct the study we considered the Portuguese version (SOPT) and have downloaded the data from the Stack Exchange (SE) data dump\footnote{https://archive.org/details/stackexchange}. The data extraction has been performed on March 7, 2016, and included data from November 2013 to February 2016. The XML files corresponding to the tags, users, and posts were transferred to a MySql database, through a R function per type of file (i.e., posts, users, and tags). 

After creating the tables few users were eliminated due to lack of data. None of these users had AccountId (i.e., user identifier for all stackExchange websites), LastAccessDate, WebsiteUrl, Location, UpVotes, DownVotes or Age. All of these users have the same display name (i.e., "a25bedc5-3d09-41b8-82fb-ea6c353d75ae"), and whenever they have a ProfileImageUrl, it is the same\footnote{https://www.gravatar.com/avatar/?s=128\&d=identicon\&r=PG\&f=1}.
These accounts were created at different times from November 2010 to February 2016. We could not come up with a plausible reason for these anonymous users having the same display name but no other data, they do not seem to have anything in common. In total 3 SOPT users have been eliminated.

We focused on Brazilian users thus to identify their location we used \texttt{countryNameManager}\footnote{https://github.com/tue-mdse/countryNameManager}. 

Consequently the locations were identified, a group of 25 students \mb{I don't know if I should indicate where are they from}\as{Please explain how they have been selected} were selected to help to search and get the email addresses from SOPT users, each of them with 500 profiles, as detailed below:
\begin{itemize}
    \item First, we started to look each user profile by \textit{userId}, i.e. \url{http://pt.stackoverflow.com/users/1919/}, where \textbf{1919} is the \textit{userId}. as{What does `` started to look each user profile'' mean?}
    \item Then, on the user profile, we looked the email address, if it was not available we checked whether the user has a GitHub account or a personal web page. 
        \begin{itemize}
            \item With Github account was possible to find out the email address below \textit{userName} if it was not on, we used a browser extension gitDiscovered\footnote{https://gitdiscovered.com/} to discover the email address or we checked his/her public activity looking for at he/she did a git command \footnote{https://git-scm.com/docs/git-push}, and then we searched the email address using a Github API \footnote{https://api.github.com/users/userName/events/public} by \textit{userName}. \as{It seems that you have used several techniques. Please separate them: at the moment I cannot follow.}
            \item With the personal web page, we searched the email address on the section ``about me'' or \textit{sobre me} in the Portuguese language.
            \item If none of the above, we used GitHub and searched by userName from SOPT, and compared profile picture, location, skills, creation date\as{Why do you need to compare the creation date?} between SOPT and search results on GitHub, and then we followed the steps above mentioned with GitHub account.    \as{State the purpose of these comparisons.}          
       \end{itemize}            
   \end{itemize} 
To ensure the accuracy of the results, we selected a random group of 15 users of 500 profiles and searched using the steps above mentioned. In the case they\as{What?} were inconsistent\as{How do you define consistency?}, we chose another one random group of 15, and if it continued we searched all users and compared with the previous outcomes.\as{I do not understand the last sentence; please rewrite.}

