\section{Interview}
\noindent In order to understand how Brazilians use the Portuguese version of Stack Overflow, we decided to conduct a semi-structured interview because in this type of interview it is possible to collect unexpected information. We interviewed 4 Brazilians developers who work in different regions of Brazil. One of these developers never used the Portuguese version of Stack Overflow, but we interviewed him just to get his point of view about the Portuguese version of Stack Overflow. All of these interviews were conducted in Portuguese then translated to English. Both the Portuguese and English versions of the interviews can be downloaded in \wt{Add the address}.

Brazil is a big country and it might have different patterns in the software development. In order to try to cover this diversity, we used the social media to call developers from the industry who would like to participate of the interview. We selected developers from Brasilia, Pernambuco, Santa Catarina and Sao Paulo; center-west, northeast, southeast, south of Brazil, respectively.

We recorded the audio of our first interview, as proposed by \cite{Seaman:1999}, however during the transcription process we realized that it was not the best approach to follow because it took too long. This process can take approximately two times the time of the original audio recordings\cite{Gerpheide:2016} or even worse, up to eight hours per hour of audio as described by Hove et al.\cite{Hove:2005}. Thus, we decided to conduct the other interviews using some instant message software like Skype\footnote{https://www.skype.com}. 

It is not the first time that an instant message tool was used to conduct an interview \cite{IgorPhdThesis}. However, they said that using a textual chat the responses might be distract during the interview, because they can be executing other activities at the same time and it could be a threat. In our case, we do not think it as a threat because they answered the questions very fast. Furthermore, all of the responses agreed to use an instant message tool instead of video chat, actually one of them said that s/he felt more comfortable using text instead of video. This approach has been discussed in the social sciences \cite{citeulike:2009, FQS175}, and they pointed out more gains then loss.


%talk about barriers \cite{Seaman:Paradise}.. Igor Thesis is about " Supporting newcomers to overcome the barriers to contribute to open source software projects "

%As previously said, we conducted semi-structured interviews. Following are the main questions we asked to developers:
During the interview we tried to make it as much informal as possible, because we think that the respondents could feel more comfortable answering questions in a "friend to friend" talk than in a formal strict interview.

We followed some of the guidelines proposed by \cite{Seaman:1999}, for instance, at the beginning of the interview, we said that the respondents would not be evaluated, therefore there were not wrong or right answers. We were very careful to make sure the respondents could feel relax. Following is the interview guide used:


\begin{enumerate}
	\item How do you use Stack Overflow?
	\begin{enumerate}
		\item Have you ever made a question in Stack Overflow? If yes, in which version?
	\end{enumerate}
	
	\item Why do you use Stack Overflow in Portuguese instead of the English version?
	
	\item Do you think that there are some specifics subjects/questions that are easier to find in the Portuguese version?
	\begin{enumerate}
		\item (In case s/he complain about the content) what do you think about creating new content? Making questions and answering them, maybe translating the content from English to Portuguese. 
	\end{enumerate}
	
	\item Did you find any question that you know the answer? Did you answer it? (If not) Why not?
	\begin{enumerate}
		\item Did it get accepted ? (If not) Why do you think it was not accepted?
	\end{enumerate}
	
	\item Do you feel motivated in help other people? (If not) what could be done to motivate you help them?
	
	\item Do you know any other people who use the Portuguese version of the Stack Overflow.
	\begin{enumerate}
		\item (If not) Why do you think you are the only one who use it?
		\item Do you think that, even people that do not speak English, are using the English version with the help of an on-line translate tool?
	\end{enumerate}
	
	\item In your opinion, what is the importance of Stack Overflow having the Portuguese version?
\end{enumerate}

\wt{Should I describe the respondents????}

%Giovanni
%Subject 1 - In our first interview, we interviewed a developer from Sao Paulo (Southeast of Brazil). The interviewed was made in , 

%Karina
%Subject 2

%Alex
%Subject 3

%Marcio
%Subject 4


\wt{Write something here} We summarized the finds:

\begin{itemize}
	\item All of them complained about the Portuguese content. They think the English version is more complete.
	\item Subject 4 is not interesting in help the community.
	\item Subject 3 and 4 said that if they had an account they would help others. Subject 3 have an account but he did not have it when he had the chance to help.
	\item They do not make the search on Stack Overflow, first they google it, then Google shows some results from Stack Overflow.
	\item Subject 4 thinks that on-line translation tool is good enough, so s/he can use the English version without any problems.
	\item Subject 2 and 4 always find a solution for their problems, so they never had to make any question on Stack Overflow.
	\item Subject 3 thinks that the Portuguese version will soon not be necessary anymore, because s/he thinks that English is essential for those who work in the IT.
	\item Subject 1 and 4 thinks that some people do not use the Portuguese version because they do not know that there is this version. Subject 2 thinks that some people do not use it, because of the poor Portuguese content.
	\item Subject 1 prefer be more active in the Portuguese version because it is new and needs more help. And Subject 3 prefer be more active in the English version because (according to her/him) this is the official language for software development.
\end{itemize}
