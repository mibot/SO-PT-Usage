\section{Interview}
\noindent In order to understand how Brazilians use the Portuguese version of StackOverflow, we decided to conduct a semi-structured interview because \wt{ I WILL EXPLAIN THE REASON ABOUT WE CHOOSE THIS KIND OF INTERVIEW} . We interviewed 4 Brazilians developers who work in different regions of Brazil. One of these developers never used the Portuguese version of StackOverflow, but we interviewed him just to get his point of view about the Portuguese version of StackOverflow. All of these interviews were conducted in Portuguese then they were translated to English. Both the Portuguese and English versions of the interviews can be downloaded in \wt{Add the address}.

Brazil is a big country and it might have different \wt{I will add some word here that I dont know it yet =)} for software development. To try to cover this diversity, we used the social media to call for developers who would like to participate of the interview. We select developers from Santa Catarina, S�o Paulo, Bras�lia and Pernambuco; south, southeast, center-west and northeast of Brazil, respectively. \wt{I will check the international names of these States}

We recorded the audio of our first interview, however during the transcription process we realized that this was not the best approach to follow because it took too long \wt{I will think in another sentence/word}, this process can take up to eight hours per hour of audio as described by Hove et al.\cite{Hove:2005}. Thus, we decided to conduct the other interviews using the Skype chat. 

It is not the first time that a instant message tool was used to conduct an interview \cite{IgorPhdThesis}. This approach has been discussed in the social sciences \cite{citeulike:2009, FQS175} \wt{I will improve this paragraph}

talk about barriers \cite{Seaman:Paradise}.. Igor Thesis is about " Supporting newcomers to overcome the barriers to contribute to open source software projects "

metodology by \cite{Gerpheide:2016}

The guidelines; there is no right/wrong answer; better record the interview. \cite{Seaman:1999}

%Giovanni
In our first interview, we interviewed a developer from Sao Paulo (Southeast of Brazil). The interviewed was made in , Subject 1

%Karina
Subject 2

%Alex
Subject 3

%Marcio
Subject 4


summary of out finds:
\wt{I will remove the names...}
\begin{itemize}
	\item All of them complained about the Portuguese content. They think the English version is more complete.
	\item Marcio are not interesting in help the community.
	\item Alex and Karina said that if they had an account they would help others. Alex have an account but he did not have it when he had the chance to help.
	\item They do not make the search on SO, first they use google, then the google send them to SO.
	\item Marcio thinks that on-line translation tool is good enough, so he can use the English version without any problems.
	\item Karina and Marcio always find a solution for their problems, so they never had to make any question on SO.
	\item Alex thinks that the Portuguese version will soon not be necessary anymore, because he thinks that English is essential for those who work in the IT.
	\item Giovanni and Marcio thinks that some people do not use the PT version because they dont know that there is a PT version. Karina thinks that some people do not use it, because of the poor PT content.
	\item Giovanni prefer be more active in the PT version because it is new and needs more help. And Alex prefer be more active in the EN version because (according to him) English is the official language for software development.
\end{itemize}


Text from Igor's Theses.. remove it

"All the interviews followed a semi-structured script and were conducted using textual based chat tools, like Google Talk. We chose this mean once the participants are used to this kind of tool for their professional and personal activities. The interviews were conducted following three different scripts, used according to the participant?s profiles. The scripts were validated during the pilot interviews and by one specialist in qualitative studies, and one specialist in Open Source Software"

"We understand that the use of textual chat as the interview means can be considered a threat. The possibility of context change and the execution of parallel activities that distract the interviewees can be a negative aspect of using this mean. The use of Instant Messengers has been discussed in the social sciences (Opdenakker, 2006; Hinchcliffe and Gavin, 2009), and they point out that there is a set of positive effects of using these tools. In our case, we chose to use this means once the participants are used to the environment (they could choose the IM that they were more used to), and electronic means are the default (and preferred) way of communication in OSS projects."



Things to write:

- The type of interview, how it was conducted, why we did this way

The Criteria we used to select people - Brazilians from industry that use Stack Overflow in Portuguese.

- show the guideline - The guidelines;
Make it clear that we said this to the interviewed: there is no right/wrong answer and if we could record the interview...

- write that all everyone that was interviewed agreed about use chat - skype -





