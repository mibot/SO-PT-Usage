\section{Survey}
The survey was designed and reported by following the recomendations provided \cite{Kitchenham2002}.
\subsection{Goal and Research Questions}

\subsection{Respondents}
Our respondents consisted of Brazilian users who have accounts in both English and Portuguese versions of the Stack Overflow website. We focused on Brazilian users because most of SO-PT users are located in Brazil.
\subsection{Survey Structure}
The survey was structured in blocks which grouped the questions into six topics:
\subsubsection{Background}  this refers to information about the person replying such as: occupation, mother tongue, Portuguese and English skills related to writing and reading, language used. This block of 5 questions.
\subsubsection{Stack Overflow in Portuguese (SO-PT)} the objective was characterize the respondents information about the platform usage. In particular we collected information concerning: time as user, type and frequency of contribution, during what software development activities they have used SO-PT, information sources used instead of SO-PT, electronic translation tools and their types, and whether the participation in SO-PT is different to the english website of Stack Overfkow (SO). This block of 8 questions.
\subsubsection{Stack Overflow in English (SO-EN)} the refers to information regarding the platform usage. The block included questions related to items such as: time as user, type and frequency of contribution, during what software development activities they have used SO-PT, information sources used instead of SO-PT, electronic translation tools and their types. This block of 7 questions.
\subsubsection{Stack Overflow usage (Portuguese vs. English)} this consisted of questions such as: ways of contributing and their frequency, factors that affect the choice of version of Stack Overflow, and content-wise quality in both websites.
\subsubsection{Skills with Stack Overflow (SO-EN and SO-PT)} this referes to information regarding skills and their types. The block included 5 questions.
\subsubsection{Barriers in Stack Overflow (SO-EN and SO-PT)} the objetive was know the main barriers to do not contribute in Stack Overflow (SO-EN and SO-PT). This barriers were found in \cite{Ford2016Paradise}. The block included 2 questions.
\subsection{Survey Design}
\mb{To address the research questions formulated}, we drew up a survey consisting of 6 blocks of questions, with 34 questions in all in two versions Portuguese and English. Some questions were not presented to all individuals, as they were determined by the responses provided to other questions (i.e., conditional ones). Each person therefore answered a maximum of 26 questions. The electronic copy of the survey is available online in English at: https://goo.gl/aKvdQY and in Portuguese at: https://goo.gl/O8Iasd.

Most of the questions were measured using nominal scales, and a few others were measured with Likert scales, they were opened and closed questions. Some of them also included a space for extra information.
\subsection{Survey Execution}
The procedure followed consisted of the following steps:
\begin{enumerate}	
	\item The survey was online from January 24 to February 10 of 2017, using Google Forms \cite{Google}.
	\item Users were invited (via email) to participate the study. 
	\item After the surveys had been collected, analyses were performed, aiming to answer the \mb{research questions}. Data analysis was based on a quantitative analysis focusing mainly on descriptive statistics and percentages of the information collected. 
\end{enumerate}