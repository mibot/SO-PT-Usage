\documentclass[conference]{IEEEtran}
\usepackage{url}
\usepackage{xcolor}
\usepackage{ifthen}
\usepackage{amssymb}
\newboolean{showcomments}
\setboolean{showcomments}{true} % toggle to show or hide comments
\ifthenelse{\boolean{showcomments}}
{\newcommand{\nb}[2]{
        \fcolorbox{gray}{yellow}{\bfseries\sffamily\scriptsize#1}
        {\sf\small$\blacktriangleright$\textit{#2}$\blacktriangleleft$}
    }
    \newcommand{\version}{\emph{\scriptsize$-$working$-$}}
}
{\newcommand{\nb}[2]{}
    \newcommand{\version}{}
}

\newcommand{\mb}[1]{\nb{Miguel}{#1}}
\newcommand{\al}[1]{\nb{Angela}{#1}}
\newcommand{\bv}[1]{\nb{Bogdan}{#1}}
\newcommand{\wt}[1]{\nb{Weslley}{#1}}
\newcommand{\as}[1]{\nb{Alexander}{#1}}

\usepackage[normalem]{ulem}

% *** GRAPHICS RELATED PACKAGES ***
%
\ifCLASSINFOpdf
  % \usepackage[pdftex]{graphicx}
  % declare the path(s) where your graphic files are
  % \graphicspath{{../pdf/}{../jpeg/}}
  % and their extensions so you won't have to specify these with
  % every instance of \includegraphics
  % \DeclareGraphicsExtensions{.pdf,.jpeg,.png}
\else
  % or other class option (dvipsone, dvipdf, if not using dvips). graphicx
  % will default to the driver specified in the system graphics.cfg if no
  % driver is specified.
  % \usepackage[dvips]{graphicx}
  % declare the path(s) where your graphic files are
  % \graphicspath{{../eps/}}
  % and their extensions so you won't have to specify these with
  % every instance of \includegraphics
  % \DeclareGraphicsExtensions{.eps}
\fi
% graphicx was written by David Carlisle and Sebastian Rahtz. It is
% required if you want graphics, photos, etc. graphicx.sty is already
% installed on most LaTeX systems. The latest version and documentation
% can be obtained at: 
% http://www.ctan.org/pkg/graphicx
% Another good source of documentation is "Using Imported Graphics in
% LaTeX2e" by Keith Reckdahl which can be found at:
% http://www.ctan.org/pkg/epslatex
%
% latex, and pdflatex in dvi mode, support graphics in encapsulated
% postscript (.eps) format. pdflatex in pdf mode supports graphics
% in .pdf, .jpeg, .png and .mps (metapost) formats. Users should ensure
% that all non-photo figures use a vector format (.eps, .pdf, .mps) and
% not a bitmapped formats (.jpeg, .png). The IEEE frowns on bitmapped formats
% which can result in "jaggedy"/blurry rendering of lines and letters as
% well as large increases in file sizes.
%
% You can find documentation about the pdfTeX application at:
% http://www.tug.org/applications/pdftex





% *** MATH PACKAGES ***
%
%\usepackage{amsmath}
% A popular package from the American Mathematical Society that provides
% many useful and powerful commands for dealing with mathematics.
%
% Note that the amsmath package sets \interdisplaylinepenalty to 10000
% thus preventing page breaks from occurring within multiline equations. Use:
%\interdisplaylinepenalty=2500
% after loading amsmath to restore such page breaks as IEEEtran.cls normally
% does. amsmath.sty is already installed on most LaTeX systems. The latest
% version and documentation can be obtained at:
% http://www.ctan.org/pkg/amsmath





% *** SPECIALIZED LIST PACKAGES ***
%
%\usepackage{algorithmic}
% algorithmic.sty was written by Peter Williams and Rogerio Brito.
% This package provides an algorithmic environment fo describing algorithms.
% You can use the algorithmic environment in-text or within a figure
% environment to provide for a floating algorithm. Do NOT use the algorithm
% floating environment provided by algorithm.sty (by the same authors) or
% algorithm2e.sty (by Christophe Fiorio) as the IEEE does not use dedicated
% algorithm float types and packages that provide these will not provide
% correct IEEE style captions. The latest version and documentation of
% algorithmic.sty can be obtained at:
% http://www.ctan.org/pkg/algorithms
% Also of interest may be the (relatively newer and more customizable)
% algorithmicx.sty package by Szasz Janos:
% http://www.ctan.org/pkg/algorithmicx




% *** ALIGNMENT PACKAGES ***
%
%\usepackage{array}
% Frank Mittelbach's and David Carlisle's array.sty patches and improves
% the standard LaTeX2e array and tabular environments to provide better
% appearance and additional user controls. As the default LaTeX2e table
% generation code is lacking to the point of almost being broken with
% respect to the quality of the end results, all users are strongly
% advised to use an enhanced (at the very least that provided by array.sty)
% set of table tools. array.sty is already installed on most systems. The
% latest version and documentation can be obtained at:
% http://www.ctan.org/pkg/array


% IEEEtran contains the IEEEeqnarray family of commands that can be used to
% generate multiline equations as well as matrices, tables, etc., of high
% quality.




% *** SUBFIGURE PACKAGES ***
%\ifCLASSOPTIONcompsoc
%  \usepackage[caption=false,font=normalsize,labelfont=sf,textfont=sf]{subfig}
%\else
%  \usepackage[caption=false,font=footnotesize]{subfig}
%\fi
% subfig.sty, written by Steven Douglas Cochran, is the modern replacement
% for subfigure.sty, the latter of which is no longer maintained and is
% incompatible with some LaTeX packages including fixltx2e. However,
% subfig.sty requires and automatically loads Axel Sommerfeldt's caption.sty
% which will override IEEEtran.cls' handling of captions and this will result
% in non-IEEE style figure/table captions. To prevent this problem, be sure
% and invoke subfig.sty's "caption=false" package option (available since
% subfig.sty version 1.3, 2005/06/28) as this is will preserve IEEEtran.cls
% handling of captions.
% Note that the Computer Society format requires a larger sans serif font
% than the serif footnote size font used in traditional IEEE formatting
% and thus the need to invoke different subfig.sty package options depending
% on whether compsoc mode has been enabled.
%
% The latest version and documentation of subfig.sty can be obtained at:
% http://www.ctan.org/pkg/subfig




% *** FLOAT PACKAGES ***
%
%\usepackage{fixltx2e}
% fixltx2e, the successor to the earlier fix2col.sty, was written by
% Frank Mittelbach and David Carlisle. This package corrects a few problems
% in the LaTeX2e kernel, the most notable of which is that in current
% LaTeX2e releases, the ordering of single and double column floats is not
% guaranteed to be preserved. Thus, an unpatched LaTeX2e can allow a
% single column figure to be placed prior to an earlier double column
% figure.
% Be aware that LaTeX2e kernels dated 2015 and later have fixltx2e.sty's
% corrections already built into the system in which case a warning will
% be issued if an attempt is made to load fixltx2e.sty as it is no longer
% needed.
% The latest version and documentation can be found at:
% http://www.ctan.org/pkg/fixltx2e


%\usepackage{stfloats}
% stfloats.sty was written by Sigitas Tolusis. This package gives LaTeX2e
% the ability to do double column floats at the bottom of the page as well
% as the top. (e.g., "\begin{figure*}[!b]" is not normally possible in
% LaTeX2e). It also provides a command:
%\fnbelowfloat
% to enable the placement of footnotes below bottom floats (the standard
% LaTeX2e kernel puts them above bottom floats). This is an invasive package
% which rewrites many portions of the LaTeX2e float routines. It may not work
% with other packages that modify the LaTeX2e float routines. The latest
% version and documentation can be obtained at:
% http://www.ctan.org/pkg/stfloats
% Do not use the stfloats baselinefloat ability as the IEEE does not allow
% \baselineskip to stretch. Authors submitting work to the IEEE should note
% that the IEEE rarely uses double column equations and that authors should try
% to avoid such use. Do not be tempted to use the cuted.sty or midfloat.sty
% packages (also by Sigitas Tolusis) as the IEEE does not format its papers in
% such ways.
% Do not attempt to use stfloats with fixltx2e as they are incompatible.
% Instead, use Morten Hogholm'a dblfloatfix which combines the features
% of both fixltx2e and stfloats:
%
% \usepackage{dblfloatfix}
% The latest version can be found at:
% http://www.ctan.org/pkg/dblfloatfix




% *** PDF, URL AND HYPERLINK PACKAGES ***
%
%\usepackage{url}
% url.sty was written by Donald Arseneau. It provides better support for
% handling and breaking URLs. url.sty is already installed on most LaTeX
% systems. The latest version and documentation can be obtained at:
% http://www.ctan.org/pkg/url
% Basically, \url{my_url_here}.




% *** Do not adjust lengths that control margins, column widths, etc. ***
% *** Do not use packages that alter fonts (such as pslatex).         ***
% There should be no need to do such things with IEEEtran.cls V1.6 and later.
% (Unless specifically asked to do so by the journal or conference you plan
% to submit to, of course. )


% correct bad hyphenation here
\hyphenation{op-tical net-works semi-conduc-tor}


\begin{document}
%
% paper title
% Titles are generally capitalized except for words such as a, an, and, as,
% at, but, by, for, in, nor, of, on, or, the, to and up, which are usually
% not capitalized unless they are the first or last word of the title.
% Linebreaks \\ can be used within to get better formatting as desired.
% Do not put math or special symbols in the title.
\title{Is Stack Overflow in Portuguese more attractive for Brazilian users?}


% author names and affiliations
% use a multiple column layout for up to three different
% affiliations
\author{
    \IEEEauthorblockN{
        Angela Lozano\IEEEauthorrefmark{1},
        Bogdan Vasilescu\IEEEauthorrefmark{2},
        Miguel Botto Tobar\IEEEauthorrefmark{3}\IEEEauthorrefmark{4}, 
        Weslley Torres\IEEEauthorrefmark{4}, and
        Alexander Serebrenik\IEEEauthorrefmark{4},~\IEEEmembership{Fellow,~IEEE}
        }
    \IEEEauthorblockA{\IEEEauthorrefmark{1}Vrije Universiteit Brussel, Brussels, Belgium}
    \IEEEauthorblockA{\IEEEauthorrefmark{2}Carniege Mellon University, Pittsburgh, PA, USA}
    \IEEEauthorblockA{\IEEEauthorrefmark{3}University of Guayaquil, Guayaquil, Ecuador}
    \IEEEauthorblockA{\IEEEauthorrefmark{4}Eindhoven University of Technology, Eindhoven, The Netherlands}% <-this % stops an unwanted space
    \thanks{Manuscript received December 1, 2012; revised September 17, 2014. 
        Corresponding author: M. Shell (email: http://www.michaelshell.org/contact.html).}
 }


% conference papers do not typically use \thanks and this command
% is locked out in conference mode. If really needed, such as for
% the acknowledgment of grants, issue a \IEEEoverridecommandlockouts
% after \documentclass

% for over three affiliations, or if they all won't fit within the width
% of the page, use this alternative format:
% 
%\author{\IEEEauthorblockN{Michael Shell\IEEEauthorrefmark{1},
%Homer Simpson\IEEEauthorrefmark{2},
%James Kirk\IEEEauthorrefmark{3}, 
%Montgomery Scott\IEEEauthorrefmark{3} and
%Eldon Tyrell\IEEEauthorrefmark{4}}
%\IEEEauthorblockA{\IEEEauthorrefmark{1}School of Electrical and Computer Engineering\\
%Georgia Institute of Technology,
%Atlanta, Georgia 30332--0250\\ Email: see http://www.michaelshell.org/contact.html}
%\IEEEauthorblockA{\IEEEauthorrefmark{2}Twentieth Century Fox, Springfield, USA\\
%Email: homer@thesimpsons.com}
%\IEEEauthorblockA{\IEEEauthorrefmark{3}Starfleet Academy, San Francisco, California 96678-2391\\
%Telephone: (800) 555--1212, Fax: (888) 555--1212}
%\IEEEauthorblockA{\IEEEauthorrefmark{4}Tyrell Inc., 123 Replicant Street, Los Angeles, California 90210--4321}}




% use for special paper notices
%\IEEEspecialpapernotice{(Invited Paper)}




% make the title area
\maketitle

% As a general rule, do not put math, special symbols or citations
% in the abstract
\begin{abstract}
The abstract goes here.
\end{abstract}

\IEEEpeerreviewmaketitle

\section{Introduction}

Software development and maintenance are activities that often involves many concepts and reference documents (citar). Many software aspects may be changed over time. In order to work with them and details involved in a software project, developers often need helps from one another. Nowadays a widely used way is for developers to ask questions and/or answer them in various online forums. StackOverflow (SO)\footnote{https://stackoverflow.com/} is a Q\&A site with more than six million registered users. SO also has the version on Portuguese, Russian and Spanish. 

The main purpose of the study was to understand the motivations behind stack overflow usage, to what extent it has/can contribute to improve skills of its users. In particular, we were interested in the motivations and profiles of users whose mother tongue is not English, and in case they are bilingual their participation (or lack of participation) in the initial website (i.e., in English \url{http://stackoverflow.com/} and in the website dedicated to their mother tongue (i.e., in Portuguese \url{http://pt.stackoverflow.com/}. The main research questions that guided our study are:
\noindent \textbf{RQ1: Which is the purpose of using SOPT?} \\ 
\mb{(asker, mostly asker, both equality, mostly answerer, answerer, no activity)} \\
\textbf{RQ2: What kinds of questions are asked on SOPT and which ones are answered?} \\ 
\mb{related to programming languages, environment, framework, so on}\\
\textbf{RQ3: What are most common problems faced related to usage?} \\ 
\mb{many questions, less answers?}\\
\as{Here you need to explain why did you decide to focus on Brazil as opposed to any other country in the world.}

\section{Methodology}
\noindent To conduct the study we considered the Portuguese version (SOPT) and have downloaded the data from the Stack Exchange (SE) data dump\footnote{https://archive.org/details/stackexchange}. The data extraction has been performed on March 7, 2016, and included data from November 2013 to February 2016. The XML files corresponding to the tags, users, and posts were transferred to a MySql database, through a R function per type of file (i.e., posts, users, and tags). 

After creating the tables few users were eliminated due to lack of data. None of these users had AccountId (i.e., user identifier for all stackExchange websites), LastAccessDate, WebsiteUrl, Location, UpVotes, DownVotes or Age. All of these users have the same display name (i.e., "a25bedc5-3d09-41b8-82fb-ea6c353d75ae"), and whenever they have a ProfileImageUrl, it is the same\footnote{https://www.gravatar.com/avatar/?s=128\&d=identicon\&r=PG\&f=1}.
These accounts were created at different times from November 2010 to February 2016. We could not come up with a plausible reason for these anonymous users having the same display name but no other data, they do not seem to have anything in common. In total 3 SOPT users have been eliminated.

We focused on Brazilian users thus to identify their location we used \texttt{countryNameManager}\footnote{https://github.com/tue-mdse/countryNameManager}. 

Consequently the locations were identified, a group of 25 students \mb{I don't know if I should indicate where are they from}\as{Please explain how they have been selected} were selected to help to search and get the email addresses from SOPT users, each of them with 500 profiles, as detailed below:
\begin{itemize}
    \item First, we started to look each user profile by \textit{userId}, i.e. \url{http://pt.stackoverflow.com/users/1919/}, where \textbf{1919} is the \textit{userId}. as{What does `` started to look each user profile'' mean?}
    \item Then, on the user profile, we looked the email address, if it was not available we checked whether the user has a GitHub account or a personal web page. 
        \begin{itemize}
            \item With Github account was possible to find out the email address below \textit{userName} if it was not on, we used a browser extension gitDiscovered\footnote{https://gitdiscovered.com/} to discover the email address or we checked his/her public activity looking for at he/she did a git command \footnote{https://git-scm.com/docs/git-push}, and then we searched the email address using a Github API \footnote{https://api.github.com/users/userName/events/public} by \textit{userName}. \as{It seems that you have used several techniques. Please separate them: at the moment I cannot follow.}
            \item With the personal web page, we searched the email address on the section ``about me'' or \textit{sobre me} in the Portuguese language.
            \item If none of the above, we used GitHub and searched by userName from SOPT, and compared profile picture, location, skills, creation date\as{Why do you need to compare the creation date?} between SOPT and search results on GitHub, and then we followed the steps above mentioned with GitHub account.    \as{State the purpose of these comparisons.}          
       \end{itemize}            
   \end{itemize} 
To ensure the accuracy of the results, we selected a random group of 15 users of 500 profiles and searched using the steps above mentioned. In the case they\as{What?} were inconsistent\as{How do you define consistency?}, we chose another one random group of 15, and if it continued we searched all users and compared with the previous outcomes.\as{I do not understand the last sentence; please rewrite.}


\section{Results}
\subsection{\mb{SO-PT Users}}
We identify the location of 7.264 users of SOPT which corresponds to 27\% of its users. As we foresaw, most of the users of SOPT are located at Portuguese-speaking countries, in particular in Brazil (see Table \ref{tbl:Locations}). Although there is a wide range of non-Portuguese speaking countries users, when looking at percentages these countries only represent 2%. 

\begin{table}[ht]
\begin{center}
\scriptsize{
\begin{tabular}{lrrr}
Country & Total \\
\hline
Brazil* 			& 5954 \\
Portugal* 			& 599 \\
United States 		& 220 \\
United Kingdom      & 80 \\
Canada				& 44 \\
France              & 25 \\
Germany				& 42 \\
India				& 30 \\
The Netherlands		& 20 \\
Mozambique*		    & 14 \\
Angola*		 		& 8 \\
Cape Verde*		    & 4 \\
Other non Portuguese countries		 & 224 \\
None 				& 19415 \\
\end{tabular}
\caption{User's location in SOPT. Portuguese speaking countries are marked with an asterisk.}
\label{tbl:Locations}
}
\end{center}
\end{table}

For half of the users whose location was identified, we could identify their gender (65\%). Females are an overwhelming minority (4\% SOPT users).

\subsection{Survey}
A total of 215 Brazilian users from the 1050 responded to the survey during the time it was online. This result is significant because of the difficulty normally involved in obtaining such a large quantity of individuals suitable for making up a target population. 

\subsubsection{Background}
The majority of the respondents (92\%) are ICT professionals, and only 8\% have other professions such as students and academics. The ICT role most frequently played by the participants is that of software engineer/developer/system or database administrator (83\%), and data scientist/machine learning developer/statistics or maths developer (3\%). The remaining roles (technical coordinator, electronics technician, head of software development, web designer, software architect, etc.) are performed by less than 6\% each. This means that ICT professionals use SO to ask/answer issues related to their sotware development activities.

We asked the respondents about mother tongue, writing and reading skills in both Portuguese and English, and language used in their daily activities. In the first question, the mother tongue of the respondents is Portuguese (99\%), and the remaining 1\% is Finnish. This result is consistent due to the official language in Brazil is the Portuguese. In the part related to writing and reading Portuguese skills, the majority of them have excellent (native) level (95\%), and the remaining 5\% have an advanced level. This is due to as we told above about the language used in Brazil. In the same way about writing and reading English skills. The level most frequently played by the participants is that of advanced (59\%), followed by intermediate (43\%), excellent (native) 13\%, and survival (level) 9\%. These results are due to most of them use English in their software development activities such as coding, debugging, etc. Finally, the part related to the language used in daily activities, Portuguese is used in their homes (84\%) and 87\% at streets, banks, post offices, etc., and Portuguese and English are used at work (62\%) and school (65\%). 

\subsubsection{Stack Overflow in Portuguese (SO-PT)}
The majority of the respondents (38\%) are users with between 1 and 2 years as members of SO-PT, followed by 33\% with less than 1 year, and 29\% with more than 3 years. This results have sense due to SO-PT was lauched in 2013 and most of its users are from SO-EN.
  
We asked the respondents about their contributions and the frequency, the contribution types asked them were: creating questions, commenting on questions, answering questions, commenting on answers, editing questions and/or answers, voting up/down in questions and/or answers. The roles most frequently played by the participants are that of both never and almost never (about once a year) in all, followed by rarely (not more than once a month) and ocassionally (about once a week); and frequently (almost) was less chose. These results reported that SO-PT users do not use the Stack Overflow in Portuguese for these types of contributions (for more details see table \ref{tbl:contributions_SO-PT}). 

\begin{table}[!htbp]
	\begin{center}		
	\begin{adjustbox}{max width=\textwidth}
	\begin{tabular}{llllll}
		& Never & \begin{tabular}[c]{@{}l@{}}Almost \\ Never\end{tabular} & Rarely & Occasionally & Frequently \\
		\hline
		\begin{tabular}[c]{@{}l@{}}Creating \\ questions\end{tabular} & 55\% & 26\% & 18\% & 1\% &  \\
		\begin{tabular}[c]{@{}l@{}}Commenting \\ on questions\end{tabular} & 35\% & 38\% & 20\% & 5\% & 2\% \\
		\begin{tabular}[c]{@{}l@{}}Answering \\ questions\end{tabular} & 35\% & 36\% & 23\% & 4\% & 2\% \\
		\begin{tabular}[c]{@{}l@{}}Commenting \\ on answers\end{tabular} & 41\% & 31\% & 23\% & 3\% & 1\% \\
		\begin{tabular}[c]{@{}l@{}}Editing \\ questions \\ and/or answers\end{tabular} & 54\% & 25\% & 16\% & 3\% & 1\% \\
		\begin{tabular}[c]{@{}l@{}}Voting up \\ questions \\ and/or answers\end{tabular} & 28\% & 24\% & 23\% & 18\% & 7\% \\
		\begin{tabular}[c]{@{}l@{}}Voting down \\ questions \\ and/or answers\end{tabular} & 39\% & 33\% & 17\% & 10\% & 1\%
	\end{tabular}
   \end{adjustbox}
\caption{Contributions in SO-PT}
\label{tbl:contributions_SO-PT}
\end{center}
\end{table}

\subsubsection{Stack Overflow in Portuguese (SO-EN)}
The majority of the respondents (40\%) are users with between 4 and 6 years as members of SO-EN, followed by 29\% with more than 6 years, and 13\% with less than 1 year. The remaining 18\% corresponds to some people did not answer all the questions (because of some conditional questions)





%\input{discussion.tex}
%\input{conclusion.tex}

% conference papers do not normally have an appendix


% use section* for acknowledgment
\section*{Acknowledgment}


The authors would like to thank...





% trigger a \newpage just before the given reference
% number - used to balance the columns on the last page
% adjust value as needed - may need to be readjusted if
% the document is modified later
%\IEEEtriggeratref{8}
% The "triggered" command can be changed if desired:
%\IEEEtriggercmd{\enlargethispage{-5in}}

% references section

% can use a bibliography generated by BibTeX as a .bbl file
% BibTeX documentation can be easily obtained at:
% http://mirror.ctan.org/biblio/bibtex/contrib/doc/
% The IEEEtran BibTeX style support page is at:
% http://www.michaelshell.org/tex/ieeetran/bibtex/
%\bibliographystyle{IEEEtran}
% argument is your BibTeX string definitions and bibliography database(s)
%\bibliography{IEEEabrv,../bib/paper}
%
% <OR> manually copy in the resultant .bbl file
% set second argument of \begin to the number of references
% (used to reserve space for the reference number labels box)
\begin{thebibliography}{1}

\bibitem{IEEEhowto:kopka}
H.~Kopka and P.~W. Daly, \emph{A Guide to \LaTeX}, 3rd~ed.\hskip 1em plus
  0.5em minus 0.4em\relax Harlow, England: Addison-Wesley, 1999.

\end{thebibliography}




% that's all folks
\end{document}


