% !TEX root = main.tex
\section{Theory}
\as{This is an attempt to position our work in the broader context}
The need of translation and, broader localization, of software elements targeting end users, such as user interfaces, user manuals and support platforms, has been commonly recognized. 
This is, however, less obvious for documents targeting software developers.
For instance, the ``Java for Consumers'' page\footnote{\url{https://www.java.com/download/}}
exists in such languages as Dutch and French, while no such counterparts exist for the ``Java for Developers'' page\footnote{\url{http://www.oracle.com/technetwork/indexes/downloads/index.html?ssSourceSiteId=ocomen}}. 
Still, Java 8 documentation has been translated, \eg to Japanese\footnote{\url{http://docs.oracle.com/javase/jp/8/docs/api/}}, documentation of PostgreSQL 9.5 into Russian\footnote{\url{https://postgrespro.ru/docs/postgresql/9.5/index.html}} and novatec is a Brazilian company specialized in translating O'Reilly books into Portuguese.
Going beyond translation, original software engineering books have been published, \eg in \as{We need examples; in Russian I can find only translations and textbooks}. Moreover, online developers' communities exists, \eg in Spanish\footnote{\url{http://www.lawebdelprogramador.com/}} and French\footnote{\url{http://www.developpez.net/forums/}}, while StackOverflow (SO)\footnote{https://stackoverflow.com/} in addition to English supports equivalent Q\&A platforms in, \eg Portuguese, Russian and Japanese. 

The question hence arises of the function of those non-English original or translated information sources in
the developers' communities. 
Do they empower developers by providing them with access to technological documentation or impair their abilities not only by not encouraging them to learn English but also by encouraging them to rely on resources in their own language that---due to the popularity of English at expense of other languages---might be scarce, erroneous or out-dated? 
Are those information sources still relevant anno 2016 despite the progress made in automatic translation? 

This discussion is clearly related to the question of the role of English as a neutral \emph{lingua franca} or as a mechanism of domination~\cite{Tardy,Ammon}. \as{Both Tardy~\cite{Tardy} and most chapters in Ammon's book~\cite{Ammon} do not discuss technology but science; still this seems to be closely related.}
\as{One of the claims in science is that papers written in languages other than English or published in non-English-speaking venues are ``invisible'', they are not cited etc. Can something similar be claimed for software engineering? I know that Lua has been created in Brazil and Python in the Netherlands. Did this somehow affect their adoption?}
Indeed, if English is seen as a necessary and neutral lingua franca, then technological solutions such as automatic 
translation should be encouraged as they have the potential of alleviating scarcity of the non-English resources or their tardiness. 
If, however, English is seen as a domination mechanism \as{here I wanted to say something like ``the developers need tools to oppose this dominance'' but then I've started doubting whether this is true.}
\as{Carmel~\cite{Carmel} explains why English is the dominant language in software.
``In addition to these nine better known U.S. competitive advantages, two
culturally linked assertions are presented that have received scant attention vis-\'{a}-vis
competitive analysis. First, the industrial evolution of software development is at an
immature stage still a cottage industry practiced by craftsmen in a cultural milieu of
artisans and thus does not track other global high-technology trends. Second, packaged
software is part of the copyright industry (e.g., film and music) in which United
States-based firms have a sustained advantage. While manufacturing capabilities are
significant for technology industries, culturally related factors, such as creativity, are
more important for copyright industries. The U.S. ``culture of software'' which helps
explain U.S. hegemony, is introduced and discussed. The three elements of this culture
are the culture of individuals as manifested by the individualistic computer hacker;
the entrepreneurial culture and its risk-taking ethos; and the software development
culture with its embrace of ad hoc, innovation-driven development as opposed to routinized,
production-driven development.}