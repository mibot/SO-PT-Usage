\section{Introduction}

Software development and maintenance are activities that often involves many concepts and reference documents. Many software aspects may be changed over time. In order to work with them and details involved in a software project, developers often need helps from one another \cite{Wang2013}. Nowadays, developers are using online forums as a widely way to ask questions and/or answer them about different issues related software development. StackOverflow (SO) \footnote{http://stackoverflow.com/} is a Q\&A site with more than six million registered users. SO also has the version on Portuguese, Russian and Spanish. 

The main purpose of the study was to understand the motivations behind stack overflow usage, to what extent it has/can contribute to improve skills of its users. In particular, we were interested in the motivations and profiles of users whose mother tongue is not English, and in case they are bilingual their participation (or lack of participation) in the initial website (i.e., in English \url{http://stackoverflow.com/} and in the website dedicated to their mother tongue (i.e., in Portuguese \url{http://pt.stackoverflow.com/}. The main research questions that guided our study are:

\noindent \textbf{RQ1: Which is the purpose of using SOPT?}
\mb{(asker, mostly asker, both equality, mostly answerer, answerer, no activity)} \\
\textbf{RQ2: What kinds of questions are asked on SOPT and which ones are answered?} \\ 
\mb{related to programming languages, environment, framework, so on}\\
\textbf{RQ3: What are most common problems faced related to usage?} \\ 
\mb{many questions, less answers?}\\

To this study, we focus on Brazilian users because they are the biggest Portuguese-speaking community on SOPT, and in addition, the Brazilian software developers have proven highly skilled in providing software solutions \mb{??}.   
