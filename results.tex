\section{Results}
\subsection{\mb{SO-PT Users}}
\noindent We identify the location of 7.264 users of SOPT which corresponds to 27\% of its users. As we foresaw, most of the users of SOPT are located at Portuguese-speaking countries, in particular in Brazil (see Table \ref{tbl:Locations}). Although there is a wide range of non-Portuguese speaking countries users, when looking at percentages these countries only represent 2%. 

\begin{table}[ht]
\begin{center}
\scriptsize{
\begin{tabular}{lrrr}
Country & Total \\
\hline
Brazil* 			& 5954 \\
Portugal* 			& 599 \\
United States 		& 220 \\
United Kingdom      & 80 \\
Canada				& 44 \\
France              & 25 \\
Germany				& 42 \\
India				& 30 \\
The Netherlands		& 20 \\
Mozambique*		    & 14 \\
Angola*		 		& 8 \\
Cape Verde*		    & 4 \\
Other non Portuguese countries		 & 224 \\
None 				& 19415 \\
\end{tabular}
\caption{User's location in SOPT. Portuguese speaking countries are marked with an asterisk.}
\label{tbl:Locations}
}
\end{center}
\end{table}

For half of the users whose location was identified, we could identify their gender (65\%). Females are an overwhelming minority (4\% SOPT users).

\subsection{Survey}
\noindent A total of 215 Brazilian users from the 1050 responded to the survey during the time it was online. This result is significant because of the difficulty normally involved in obtaining such a large quantity of individuals suitable for making up a target population. 

\subsubsection{Background}
\noindent The majority of the respondents (92\%) are ICT professionals, and only 8\% have other professions such as students and academics. The ICT role most frequently played by the participants is that of software engineer/developer/system or database administrator (83\%), and data scientist/machine learning developer/statistics or maths developer (3\%). The remaining roles (technical coordinator, electronics technician, head of software development, web designer, software architect, etc.) are performed by less than 6\% each. This fact indicated that the ICT professionals use Stack Overflow to ask/answer issues related to their software development activities.

We asked the respondents subjects related to: mother tongue, writing and reading skills in both Portuguese and English, and language used in their daily activities. The mother tongue of the respondents is Portuguese (99\%), and the remaining 1\% is Finnish. Regarding writing and reading Portuguese skills, a significant percent of them have an excellent (native) level (94\% in writing) and (96\% in reading), and the remaining 6\% in writing and 4\% in reading have an advanced level. These results were coherent due to the official language in Brazil is the Portuguese. In the same way about writing and reading English skills. The level most frequently played by the participants is that of advanced (52\% in writing) and (67\% in reading), followed by intermediate (35\% in writing) and (17\% in reading), excellent (native) 13\% and 6\% respectively, and survival (level) 7\% and 3\% respectively. This fact indicated that an average of respondents uses English in their software development activities. Finally, a considerable part of the respondents use Portuguese in their homes (84\%) and 87\% at streets, banks, post offices, etc.; and Portuguese and English are used at work (62\%) and school (65\%). \mb{In general terms, a large proportion of them uses their mother tongue because they feel more comfortable using it, but another group considers that they need to use both Portuguese and English due to they could be part of international software development companies and/or they could receive a bilingual education in their schools/universities} asuumption (see table \ref{tbl:languageSkills}).

\begin{table}[!htbp]
    \centering    
    \begin{adjustbox}{max width=\textwidth}
    \begin{tabular}{l|c|c|c|c}
        & \multicolumn{2}{c}{Portuguese} & \multicolumn{2}{|c}{English} \\
        & Writing & Reading & Writing & Reading \\
        \hline
        Survival level & - & - & 7\% & 3\% \\
        Intermediate & - & - & 35\% & 17\% \\
        Advanced & 6\% & 4\% & 52\% & 67\% \\
        \begin{tabular}[c]{@{}c@{}}Excellent\\(Native)\end{tabular} & 94\% & 96\% & 6\% & 14\%
    \end{tabular}
     \end{adjustbox}
     \caption{Language Skills in SO-PT}
     \label{tbl:languageSkills}
\end{table}

\subsubsection{Stack Overflow in Portuguese (SO-PT)}
\noindent The 38\% of the respondents are users with between 1 and 2 years as members of SO-PT, followed by 33\% with less than 1 year, and 29\% with more than 3 years. This results have sense due to SO-PT was launched in 2013 and most of its users are from SO-EN.
  
Regarding contributions and frequency, the contribution types asked of the respondents were: creating questions, commenting on questions, answering questions, commenting on answers, editing questions and/or answers, voting up/down in questions and/or answers. The roles most frequently played by the participants are that of both never and almost never (about once a year) in all, followed by rarely (not more than once a month) and ocassionally (about once a week); and frequently (almost) was less chosen. This result reported that SO-PT users do not use the Stack Overflow in Portuguese for these types of contributions, in other words, they just read/surf in SO-PT when they are stuck with software development issues in order to solve it. (for more details see table \ref{tbl:contributions_SO-PT}) \mb{graph}. 

\begin{table}[!htbp]
	\begin{center}		
	\begin{adjustbox}{max width=\textwidth}
	\begin{tabular}{l|l|l|l|l|l}
		& Never & \begin{tabular}[|c]{@{}l@{}}Almost\\Never\end{tabular} & Rarely & Occasionally & Frequently \\
		\hline
		\begin{tabular}[c]{@{}l@{}}Creating \\ questions\end{tabular} & 55\% & 26\% & 18\% & 1\% &  \\
		\begin{tabular}[c]{@{}l@{}}Commenting \\ on questions\end{tabular} & 35\% & 38\% & 20\% & 5\% & 2\% \\
		\begin{tabular}[c]{@{}l@{}}Answering \\ questions\end{tabular} & 35\% & 36\% & 23\% & 4\% & 2\% \\
		\begin{tabular}[c]{@{}l@{}}Commenting \\ on answers\end{tabular} & 41\% & 31\% & 23\% & 3\% & 1\% \\
		\begin{tabular}[c]{@{}l@{}}Editing \\ questions \\ and/or answers\end{tabular} & 54\% & 25\% & 16\% & 3\% & 1\% \\
		\begin{tabular}[c]{@{}l@{}}Voting up \\ questions \\ and/or answers\end{tabular} & 28\% & 24\% & 23\% & 18\% & 7\% \\
		\begin{tabular}[c]{@{}l@{}}Voting down \\ questions \\ and/or answers\end{tabular} & 39\% & 33\% & 17\% & 10\% & 1\%
	\end{tabular}
   \end{adjustbox}
\caption{Contributions in SO-PT}
\label{tbl:contributions_SO-PT}
\end{center}
\end{table}

\subsubsection{Stack Overflow in Portuguese (SO-EN)}
\noindent The majority of the respondents (40\%) are users with between 4 and 6 years as members of SO-EN, followed by 29\% with more than 6 years, and 13\% with less than 1 year. The remaining 18\% corresponds to some people did not answer this block of questions because we asked if their participation in SO-PT is different to the English website.

The role most frequently played by the participants is that of rarely (not more than once a month), followed by almost never (about once a year), occasionally (about once a week) and never; and the role with less respondents was frequently (almost daily). These results reported that SO-PT users do not use the Stack Overflow in Portuguese for these types of contributions (for more details see table \ref{tbl:contributions_SO-PT}) \mb{graph}. This fact reported that SO-PT users use the Stack Overflow in English more than SO-PT, however, their contributions are limited due to faced with language problems (most of them are non-speaking Portuguese) (for more details see table \ref{tbl:contributions_SO-PT}) \mb{graph}.  

\begin{table}[!htbp]
    \begin{center}		
        \begin{adjustbox}{max width=\textwidth}
            \begin{tabular}{|l|l|l|l|l|l}
                & Never & \begin{tabular}[|c]{@{}l@{}}Almost\\Never\end{tabular} & Rarely & Occasionally & Frequently \\
                \hline
                \begin{tabular}[c]{@{}l@{}}Creating \\ questions\end{tabular} & 13\% & 35\% & 29\% & 5\% & \\
                \begin{tabular}[c]{@{}l@{}}Commenting \\ on questions\end{tabular} & 13\% & 21\% & 32\% & 14\% & 2\% \\
                \begin{tabular}[c]{@{}l@{}}Answering \\ questions\end{tabular} & 13\% & 24\% & 34\% & 10\% & 1\% \\
                \begin{tabular}[c]{@{}l@{}}Commenting \\ on answers\end{tabular} & 13\% & 23\% & 30\% & 15\% & 1\% \\
                \begin{tabular}[c]{@{}l@{}}Editing \\ questions \\ and/or answers\end{tabular} & 26\% & 17\% & 30\% & 9\% & 1\% \\
                \begin{tabular}[c]{@{}l@{}}Voting up \\ questions \\ and/or answers\end{tabular} & 8\% & 4\% & 24\% & 27\% & 20\% \\
                \begin{tabular}[c]{@{}l@{}}Voting down \\ questions \\ and/or answers\end{tabular} & 16\% & 14\% & 27\% & 18\% & 7\%
            \end{tabular}
        \end{adjustbox}
        \caption{Contributions in SO-EN}
        \label{tbl:contributions_SO-EN}
    \end{center}
\end{table}

\subsubsection{Stack Overflow usage (Portuguese vs. English)}
\noindent We asked the respondents about the ways of contributing: SO-EN was selected for a big percent of them as the website for creating questions and editing questions and/or answers. On the other hand, SO-PT was less chosen for this kind of contributions. Regarding for answering questions, commenting questions and answers, voting up/down questions and/or answer, the respondents answered they use more often SO-EN and with a small percent SO-PT. (see table \ref{tbl:way-contributions}).

\begin{table}[!htbp]
	\begin{center}		
		\begin{adjustbox}{max width=\textwidth}
			\begin{tabular}{l|c|c|c|c|c}
				& \begin{tabular}[c]{@{}c@{}}Only \\ in \\ SO-PT\end{tabular} & \begin{tabular}[c]{@{}c@{}}More\\ often \\ in \\ SO-PT\end{tabular} & \begin{tabular}[c]{@{}c@{}}Both\\ equally\end{tabular} & \begin{tabular}[c]{@{}c@{}}More\\ often\\ in \\ SO-EN\end{tabular} & \begin{tabular}[c]{@{}c@{}}Only\\ in\\ SO-EN\end{tabular} \\
				\hline
				\begin{tabular}[c]{@{}l@{}}Creating \\ questions\end{tabular} & 7\% & 8\% & 20\% & 24\% & 40\% \\
				\begin{tabular}[c]{@{}l@{}}Commenting \\ on questions\end{tabular} & 7\% & 12\% & 25\% & 31\% & 26\% \\
				\begin{tabular}[c]{@{}l@{}}Answering \\ questions\end{tabular} & 7\% & 18\% & 19\% & 31\% & 25\% \\
				\begin{tabular}[c]{@{}l@{}}Commenting \\ on answers\end{tabular} & 6\% & 14\% & 25\% & 31\% & 25\% \\
				\begin{tabular}[c]{@{}l@{}}Editing \\ questions \\ and/or answers\end{tabular} & 8\% & 9\% & 28\% & 25\% & 29\% \\
				\begin{tabular}[c]{@{}l@{}}Voting up \\ questions \\ and/or answers\end{tabular} & 4\% & 5\% & 27\% & 40\% & 24\% \\
				\begin{tabular}[c]{@{}l@{}}Voting down \\ questions \\ and/or answers\end{tabular} & 5\% & 4\% & 33\% & 33\% & 25\%
			\end{tabular}
		\end{adjustbox}
		\caption{SO-PT vs. SO-EN}
		\label{tbl:way-contributions}
	\end{center}
\end{table}


